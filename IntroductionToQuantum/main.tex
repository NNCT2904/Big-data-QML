\section{Introduction to Quantum Computing} \label{Sec: Introduction to Quantum Computing}

\href{https://www.sri.com/story/a-brief-introduction-to-quantum-computing/#:~:text=Quantum%20computing%20is%20generally%20viewed,the%20atomic%20and%20subatomic%20level.}{brief history of quantum computing}

While the term \emph{Quantum Computing} has only began to be the topic in the public eyes, \emph{Quantum Information Science and Mechanics} has aged several decades.
\emph{Quantum mechanics} is the field of study of nature at the smallest scale.
1990s marked the begining of quantum mechanics by many physicists, e.g. Albert Einstein, Paul Dirac, Erwin Schrodinger among others.
Until now, their contribution to the fiels still chimes in many new theories and experiments.

In the 80s, Richard Feyman has marked the beginning of Quantum Computers with his envision of a new way of computation based on quantum theories \cite{feynmanSimulatingPhysicsComputers1982}, which can handle problems that are not solvable by a classical computer.

Classical computers can perform calculations based on bits of values one and zero. 
Instead, the qubit (quantum bit) can have the property of classical bits, or it can take both value of one or zero at the same time, thanks to the ``superposition'' property of quantum systems.
Theoretically, qubits can achieve an exponential higher information density compared to qubits, with $n$ qubits we can expect to store $2^n$ quantum states, compared to classical bit with $n$ states.

In practical, the stage of technological development of quantum hardware at the moment is in the early days, comparable to the very first computer models a decade ago.
At this stage their applications are limited, and quantum systems nowadays are often falls into the scope of Noisy Intermediate-Scale Quantum (NISQ) Computer \cite{brooksQuantumSupremacyHunt2019}.
Qubits are very vulnerable to environmental noise, decoherence, and gate control precision issues; we are simply not having enough qubits to perform autocorrection codes.