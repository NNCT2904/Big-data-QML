\section{Introduction to Machine Learning} \label{Sec: Introduction to Machine Learning}

\href{https://blog.paperspace.com/beginners-guide-to-quantum-machine-learning/}{Beginner's Guide to Quantum Machine Learning}

\cite{kaurIntroductionMachineLearning2021}

Machine learning is a field of study that enable computers to learn patterns without being specifically programmed.

Machine learning algorithms build and train models based on sample data, to make decision or prediction on the new data based on past observations \cite{kozaAutomatedDesignBoth1996}.
Machine learning algorithms are widely used in many applications, including but not limited to computer vision, nature langguage processing, voice recognition and big dats \cite{khanMachineLearningComputer2020, zhangNaturalLanguageProcessing2021,tandelVoiceRecognitionVoice2020,elbouchefryLearningBigData2020, shindeReviewMachineLearning2018}.

While we have observed the reacent bloom of machine learning perfomance with GPT-4 in nature language processing \cite{openaiGPT4TechnicalReport2023}, on a lesser-know field of study, there have been attempts to implement various machine learning algorithms on NISQ devices.
Such hybrid quantum algorithms are called \emph{Variational Quantum Algorithms} (or VQA) \cite{cerezoVariationalQuantumAlgorithms2021}.

Some QML based on VQA: \cite{zoufalVariationalQuantumBoltzmann2021a, tillyVariationalQuantumEigensolver2021a}

We have explained the exponential information density in Section \ref{Sec: Superposition} and Section \ref{Sec: Entanglement}.
The same principals applies when it comes to Quantum machine learning, the quantum state of $n$ qubits in our model is a $2^n$ dimensional complex vector space.