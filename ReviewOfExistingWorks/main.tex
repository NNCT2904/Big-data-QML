\section{Existing Research and Techniques} \label{Sec: Existing Research and Techniques}

\begin{framed}
    \href{https://arxiv.org/pdf/1307.0471.pdf}{Quantum support vector machine for big data classification}

    \href{https://link.springer.com/chapter/10.1007/978-3-319-63639-9_4}{Quantum Computing and Cryptography: An Overview}

    \href{https://isg-one.com/articles/quantum-computing-and-the-future-of-big-data}{Quantum Computing and The Future of Big Data}

    \href{https://ieeexplore.ieee.org/document/7876324}{Quantum Computing in Big Data Analytics: A Survey}

\end{framed}

The current development progress of quantum machine learnign algorithms is rather limited due to hardware contrain.
However, experiments on quantum hardware can still implement several methods for generalising data patterns.
This section reviews the current growth of qml development, for example, Variational Quantum Eigensolver, Quantum Support Vector machine, Quantum Principal Component Analytics.

\subsection{Variational Quantum Eigensolver} \label{Sec: VQE}

\subsection{Quantum Support Vector Machine} \label{Sec: QSVM}

\subsection{Quantum Principal Component Analytics} \label{Sec: QPCA}
